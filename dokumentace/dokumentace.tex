\input ucwmac
\input ucw-ofs
\input ucw-verb
\input ucw-link

\def\heading#1{\goodbreak\bigskip\noindent{\csc\settextsize{15}#1}\medskip}

\centerline{\csc\settextsize{25} CampPlanner -- dokumentace}
\bigskip
\centerline{\csc\settextsize{15} Jiří Setnička}
\bigskip

CampPlanner je systém pro plánování časového rozvrhu a sledování potřebného
materiálu a úkolů při plánování akcí typu tábor či soustředění. Je realizován
jako CGI modul v jazyce Perl a jeho rozhraním je tedy webová aplikace.

Centrálním bodem CampPlanneru je samotný rozvrh, ve kterém se dají přidávat,
editovat či odebírat jednotlivé položky rozvrhu (hry, přednášky či jiné aktivity,
dále referováno jen jako {\I hry}). U každé hry je možné nastavit několik
parametrů (podle kterých se poté zobrazuje v rozvrhu či se dá filtrovat) a také
přidat popis (v současné verzi formátový pomocí Markdown syntaxe).

Umístěním do rozvrhu vznikají jednotlivé časové sloty hry. Existují tři různé
typy časových slotů: primární slot, kdy se hra přímo odehrává, a sloty na
přípravu a úklid hry. Ke každému typu časového slotu lze přiřazovat organizátory
a podle tohoto přiřazení pak lze pro každého z organizátorů zobrazit jeho
individuální rozvrh.

Posledními věcmi, které lze přidávat ke hrám, jsou úkoly a potřebný materiál.
Materiál i jednotlivé úkoly lze označovat za splněné (obstarané) a oboje lze
vypsat jako přehled napříč všemi hrami.

Celý systém je vybaven systémem pro sledování změn. Je to odlehčená varianta
plnohodnotných verzovacích systémů, nelze se totiž vracet do minulosti a slouží
tedy jen ke generování emailů s diffem oproti předchozí uložené verzi. Je
realizován tak, že ke každé hře se generuje její výpis do textového formátu,
tento se pak ukládá s každým committem do databáze a při uložení nové verze se
provede diff mezi poslední verzí uloženou v databázi a současným stavem.

\heading{Použití}

CampPlanner v současnosti používá několik dalších Perlových modulů specifických
pro potřeby Korespondenčního semináře z programování, jedná se hlavně o moduly
{\tt UCW::CGI}, {\tt KSP::CGI} a {\tt KSP:DB}, z čehož poslední dva jmenované se
dají při troše práce vyměnit za jiné (zmiňované moduly poskytují přesměrování
a přístup k databázi).

Pro použití tedy stačí CampPlanneru poskytnout připojení k databázi (tabulky
databáze v příloze~A) a inicializovat ho následující posloupností příkazů:
\verbatim{
my $planovac = new KSP::CampPlanner('sksp2014');
$planovac->setDays('2014-09-13', 8);
$planovac->prepare($user);
start_html();
$planovac->show();
}
kde \verb|start_html()| je vypsání HTML layoutu stránky před obsahem vypisovaným
CampPlannerem ({\tt CampPlanner::prepare()} totiž obsahuje kód, který ještě může
stránku přesměrovat, a proto je nutné aby zahájení výpisu HTML bylo až po něm).
Proměnná \verb|$user| pak obsahuje hash představujcí přihlášeného uživatele
a obsahující následující položky: \verb|{id: 1, name: "Jméno uživatele"}|.

\heading{Nastavení}

V základní ukázce použití již bylo zmíněno volání \verb|CampPlanner->setDays()|,
které přijímá dva parametry: počáteční datum a délku akce ve dnech (včetně
počátečního dne). Toto nastavení je nutné a bez něj se tato instance
CampPlanneru odmítne spustit.

Další nastavení je možné pomocí volání \verb|CampPlanner->setTimeslots()|,
kterému lze předat hash jednotlivých částí dne (ve formátu klíč a název každé
části dne -- je navíc nutné aby jednotlivé klíče byly pojmenované tak, aby po
jejich lexikografickém seřazení odpovídaly zamýšlenému pořadí částí dne). Pokud
se toto nenastaví, použije se dělení dne na osm částí (ráno až noc, viz zdrojové
kódy).

Ostatní nastavení (například emailová adresa na kterou se mají posílat diffy) je
pak přímo součástí zdrojového kódu.

\eject
\heading{\settextsize{20}Programátorská dokumentace}
\heading{Základní členění}

Celý modul je dle schématu MVC rozdělen do tří souborů: základní {\tt CampPlanner}
a poté {\tt CampPlanner::View} starající se o vypisování HTML
a {\tt CampPlanner::Versions} starající se o verzování.

Hlavní soubor se stará o přijmutí všech HTTP GET a POST požadavků a podle nich
pak rozhoduje o další akci. Akce může být buď odpovídající zpracování dat,
modifikace databáze a přesměrování (typická reakce při HTTP POST požadavcích),
nebo zavolání odpovídající funkce z {\tt CampPlanner::View}, která se postará
o vypsání dat jí předaných.

Z hlediska přístupu k databázi byl zvolen přístup zpracování většiny věcí na
straně Perlu. Modul se zeptá databáze na všechna data související s danou akcí
a na nich pak všechny další operace provádí sám (namísto zatěžování databáze
sérií joinovaných dotazů). Tento postup byl zvolen z několika důvodů.

Zaprvé, s ohledem na to, že jedna akce nebude v databázi obsahovat příliš mnoho
záznamů, aby byl výkonový rozdíl zpracování v databázi a na straně modulu
rozpoznatelný. Naopak akcí v databázi bude s časem přibývat a při mnoho akcích v
databázi by nějaký komplikovaný join přes celou tabulku (kde bude většina dat k
jiným akcím) představoval zbytečnou zátěž databáze. Druhý důvod byl, že v Perlu
se mnohem snáze popíší některé vztahy, omezení nebo spojení různých dat.

\heading{CampPlanner}

Jak bylo popsáno výše, jedná se o hlavní logickou část celé aplikace. Jeho
fungování by se dalo rozdělit do čtyř základních částí:
\itemize\ibull
\advance\hsize by 9mm
\: Načtení parametrů, načtení databáze -- funkce \verb|prepare()|, \verb|loadDatabase()|
a přidružené pomocné funkce. Na zpracování parametrů se používá externí
funkce \verb|UCW::CGI::parse_args| a všechny načítané parametry jsou uvedeny na
vršku funkce \verb|prepare()|. Zde se rovněž provádí jejich kontrola a je tak
zabráněno propašování nežádoucích hodnot dovnitř modulu.
\: Zpracování parametrů a případný update databáze a přesměrování -- stále ve
funkci \verb|prepare()|.
\: Kontrola změny ve hrách: s pomocí \verb|CampPlanner::Versions| jsou zkontrolovány
modifikované hry a případně jim je nastaven flag v databázi. Toto se provádí proto,
aby nebylo nutné provádět diff vůči databázi při každém zobrazení rozvrhu, ale
aby stačilo provést diff jen ve chvíli, kdy mohlo dojít k nějaké změně.
\: Zobrazení odpovídající části uživatelského rozhraní pomocí \verb|CampPlanner::View|
-- ve funkci \verb|show()|.
\endlist

\heading{CampPlanner::View}

Tato část modulu se stará o vypsání jí předaných dat. Sestává se z několika
funkcí starajících se vždy o výpis jednoho pohledu uživatelského rozhraní. Možné
pohledy jsou:
\itemize\ibull
\advance\hsize by 9mm
\: Rozvrh her -- Ve formátu tabulky se spojováním navazujících slotů her pro větší
přehlednost. Lze filtrovat rozvrh jen pro vybraného organizátora nebo zvýraznit
sloty jen vybrané hry.
\: Rozvrh volných organizátorů -- CampPlanner si udělá seznam všech použitých
jmen organizátorů, jejich alokací ke slotům a na základě toho v každém slotu
zobrazí organizátory, kteří nejsou přiděleni k žádné hře, úklidu ani přípravě.
\: Editace rozvrhu her -- Existuje ve dvou verzích: Ve fallback verzi (bez
nutnosti použití javascriptu) má formu editace slotů jen pro jednu konkrétní hru
pomocí checkboxů. V plné verzi se editace provádí pro všechny hry najednou
pomocí jejich přetahování Drag\&Drop metodou (viz externí knihovny níže).
\: Detail hry -- Zobrazuje podrobnosti hry a umožňuje přidávat organizátory do
slotů hry či je zase odebírat a to samé pro materiály a úkoly dané hry.
\: Editace hry -- Editace popisu a podrobností. Pro zápis popisu se používá
Markdown syntaxe a při použití javascriptu je zobrazen pomocný editor s možností
přímého náhledu.
\: Seznam her, materiálů a úkolů -- Seznamy s možností filtrování.
\endlist

\heading{Externí JS knihovny a styly}

Pro usnadnění práce s uživatelským rozhraním jsou použity následující JS knihovny:
\itemize\ibull
\advance\hsize by 9mm
\: jQuery -- Obecný pomocník pro ostatní knihovny (\hurl{http://jquery.com/})
\: EpicEditor -- Vložitelný javascriptový Markdown editor. Umožňuje hlavně online
náhled vygenerovaného HTML bez nutnosti neustálé komunikace se serverem. Pracuje
v normálním módu (kde se náhled přepíná kliknutím na ikonu) a také ve fullscreen
módu, kde se náhled HTML zobrazuje na jedné polovině obrazovky (\hurl{http://epiceditor.com/}).
\: REDIPS Drag -- Knihovna poskytující přizpůsobitelný Drag\&Drop interface pro
přetahování položek mezi buňkami tabulky. Využívá se pro snadnou a rychlou
editaci her v rozvrhu (\hurl{http://www.redips.net/javascript/drag-and-drop-table-content/}).
\endlist

Další pomocné soubory pak jsou {\tt camp\_planner.js} a {\tt camp\_planner.css}
obsahující volání výše zmíněných JS knihoven respektive styly pro zobrazení
celého CampPlanneru.

\vfill\eject
\heading{Příloha A -- Databáze}

\verbatim{
CREATE TABLE camp_planner (
	plan_id INT NOT NULL AUTO_INCREMENT PRIMARY KEY,
	camp VARCHAR(10),		-- Identifikace akce, napriklad "jksp2012"
	type ENUM('game','lecture','other'),
	name VARCHAR(255),
	description TEXT,
	place ENUM('inside','outside'),
	chosen BOOL,			-- Je hra vybrana? (pro zobrazeni u rozvrhu)
	modified BOOL			-- Jsou zde nejake necommitnute zmeny?
);
CREATE TABLE camp_planner_commits (
	commit_id INT NOT NULL,
	plan_id INT NOT NULL,
	uid INT NOT NULL,
	commit TEXT,
	commit_message TEXT,
	time DATETIME,
	FOREIGN KEY (plan_id) REFERENCES camp_planner(plan_id) ON DELETE CASCADE,
	FOREIGN KEY (uid) REFERENCES logins(uid), -- Odkaz na tabulku uživatelů
	UNIQUE (commit_id,plan_id)
);
CREATE TABLE camp_planner_tags (
	plan_id INT NOT NULL,
	tag VARCHAR(20),
	FOREIGN KEY (plan_id) REFERENCES camp_planner(plan_id) ON DELETE CASCADE,
	UNIQUE (plan_id,tag)
);
CREATE TABLE camp_planner_timeslots (
	plan_id INT NOT NULL,
	type ENUM('primary','preparation','cleanup'),
	timeslot VARCHAR(30),		-- Identifikace slotu
	FOREIGN KEY (plan_id) REFERENCES camp_planner(plan_id) ON DELETE CASCADE,
	UNIQUE (plan_id,type,timeslot)
);
CREATE TABLE camp_planner_orgs (
	plan_id INT NOT NULL,
	type ENUM('garant','assistant','primary','preparation','cleanup'),
	orgname VARCHAR(30),
	FOREIGN KEY (plan_id) REFERENCES camp_planner(plan_id) ON DELETE CASCADE,
	UNIQUE (plan_id,type,orgname)
);
CREATE TABLE camp_planner_materials (
	material_id INT NOT NULL AUTO_INCREMENT PRIMARY KEY,
	plan_id INT NOT NULL,
	name VARCHAR(255),
	note TEXT,
	completed BOOL,
	FOREIGN KEY (plan_id) REFERENCES camp_planner(plan_id) ON DELETE CASCADE
);
CREATE TABLE camp_planner_todos (
	todo_id INT NOT NULL AUTO_INCREMENT PRIMARY KEY,
	plan_id INT NOT NULL,
	text TEXT,
	completed BOOL,
	FOREIGN KEY (plan_id) REFERENCES camp_planner(plan_id) ON DELETE CASCADE
);
}

\bye
